% !Mode:: "TeX:UTF-8"

\chapter{}
\textbf{
7.考虑郊野环境中的移动通信场景,给定$n$个基站的位置,它们由平面上的点$b_1,b_2,...,b_n$来指定,以及$n$个手机用户的位置,它们也指定为平面上的点$p_1,p_2,...,p_n$,最后,给定一个域参数$\Delta>0$。如果能以下述这样的方式把每个电话分配给一个基站,我们就说这组便携式电话是完全联通的。
}

\textbf{每个手机被分到不同的基站,且如果位于$p_i$的电话被分配到位于$b_j$的基站,那么在点$p_i$和$b_j$之间的直线距离至多是$\Delta$。}

\textbf{假设在点$p_i$的用户决定开车向东经过$z$距离,需要修改手机对基站的分配(可能要几次)以便保持完全连通,给出一个多项式时间算法。(假定在此期间其他手机固定。)如果可能,报告一个电话到基站的分配序列;如果不可能,报告使得完全连通性不可能在维持的一个点。算法运行时间在$O(n^3)$时间。}


