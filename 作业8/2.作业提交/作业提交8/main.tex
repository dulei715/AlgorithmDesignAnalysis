% !Mode:: "TeX:UTF-8"
\def\usewhat{dvipdfmx}                              % 定义编译方式 dvipdfmx 或者 pdflatex ,默认为 dvipdfmx
                                                    % 方式编译,如果需要修改,只需改变花括号中的内容即可。
%\setlength{\baselineskip}{20pt}
%\setlength{\headheight}{25pt}
\documentclass[a4paper,12.5pt,openany,twoside]{book}

                                           % 如果论文超过60 页 可以使用twoside 双面打印
\input{setup/package}                      % 定义本文所使用宏包
\graphicspath{{figures/}}                  % 定义所有的.eps 文件在figures 子目录下


\begin{document}                           % 开始全文
\begin{CJK*}{UTF8}{song}                   % 开始中文字体使用
\input{setup/format}                       % 完成对论文各个部分格式的设置
\frontmatter                               % 以下是论文导言部分,包括论文的封面,中英文摘要和中文目录
%\input{preface/cover}                      % 封面


%%%%%%%%%% 正文部分内容  %%%%%%%%%%
\mainmatter\defaultfont\sloppy\raggedbottom

\setlength{\intextsep}{2pt}
%\renewcommand{\ALC@linenosize}{\xiaosi}
\renewcommand\arraystretch{1.5}
\setlength{\abovecaptionskip}{2pt}
\setlength{\belowcaptionskip}{2pt}
\hfuzz=\maxdimen
\tolerance=10000
\hbadness=10000



%% !Mode:: "TeX:UTF-8"

\chapter{}
\textbf{
Let $f(n)$ and $g(n)$ be asymptotically positive functions. Prove or disprove each of the following conjectures.
\begin{enumerate}
  \item $f(n)=O(g(n))$ implies $g(n)=O(f(n))$.
  \item $f(n)+g(n)=\Theta(min(f(n),g(n)))$.
  \item $f(n)=O(g(n))$ implies $lg(f(n))=O(lg(g(n)))$, where $lg(g(n))\geq 1$ and $f(n)\geq 1$ for all sufficiently large $n$.
  \item $f(n)=O(g(n))$ implies $2^{f(n)}=O(2^{g(n)})$.
  \item $f(n)=O((f(n))^2)$.
  \item $f(n)=O(g(n))$ implies $g(n)=\Omega(f(n))$.
  \item $f(n)=\Theta(f(n/2))$.
\end{enumerate}
}

\hspace*{\fill} \\

1. \textbf{False}. Given an example that $f(n)=n$ and $g(n)=n^2$, we have $f(n)=O(g(n))$ but $g(n)\neq O(f(n))$.

2. \textbf{False}. Given an instance that $f(n)=n$ and $g(n)=n^2$, we have $f(n)+g(n)=n^2+n$. Because $\Theta(min(f(n),g(n)))=\Theta(n)$, we can conclude that $f(n)+g(n)\neq\Theta(min(f(n),g(n)))$.

3. \textbf{True}. Due to the condition that $f(n)=O(g(n))$, we can find a positive number $N$ and $C_1$ satisfy that when $n>N$, $f(n)\leq C_1\cdot g(n)$. Then we have $lg(f(n))\leq lg(C_1\cdot g(n))=lg(C_1)+lg(g(n))\leq C_2\cdot lg(g(n))$, where $C_2$ is a positive number and $C_2 > 1$.
Therefore, we have $lg(f(n))=O(lg(g(n)))$.

4. \textbf{False}. Given an example that $f(n)=2log(n)$ and $g(n)=log(n)$, we have $f(n)=O(g(n))$. $2^{f(n)}=n^2$, $2^{g(n)}=n$. Because $n^2\neq O(n)$, we can conclude that $2^{f(n)}\neq O(2^{g(n)})$.

5. \textbf{False}. Suppose that $f(n)=\frac{1}{n}$, then $(f(n))^2=\frac{1}{n^2}$. Because $\frac{1}{n}\neq O(\frac{1}{n^2})$, we can conclude that $f(n)\neq O((f(n))^2)$. 

6. \textbf{True}. Due to the condition that $f(n)=O(g(n))$, we can find a positive number $N$ and $C_1$ satisfy that when $n>N$, $f(n)\leq C_1\cdot g(n)$. That is we can find a positive number $C_2=\frac{1}{C_1}$ satisfy that when $n>N$, $g(n)\geq C_2\cdot g(n)$. Therefore we have $g(n)=\Omega(f(n))$.

7. \textbf{False}. Suppose that $f(n)=2^{2n}$, then we have $f(n/2)=2^{n}$. It is obviously that $2^{2n}\neq O(2^n)$. That is $f(n)\neq\Theta(f(n/2))$.
%% !Mode:: "TeX:UTF-8"

\chapter{}
\textbf{
Suppose you have algorithms with the six running times listed below.(Assume these are the exact number of operations performed as a function of the input size \emph{n}.) Suppose you have a computer that can perform $10^{10}$ operations per second, and you need to compute a result in at most an hour of computation. For each of the algorithms, what is the largest input size \emph{n} for which you would be able to get the result within an hour?
}

\textbf{(a) $n^2$}

\textbf{(b) $n^3$}

\textbf{(c) $100n^2$}

\textbf{(d) $n\textrm{log}n$}

\textbf{(e) $2^n$}

\textbf{(f) $2^{2^{n}}$}


\hspace*{\fill} \\

{
    Let $X\in\{(a),(b),(c),(d),(e),(f)\}$ be one of the six running times. We have $X\leq 3600\times 10^{10}$. We calculate the result in Table~\ref{limited_n}.
    
\begin{table}[H]
\caption{Limited value of n}
\label{limited_n}
\centering
\scalebox{1}{
\begin{tabular}{c|c|c|c|c|}
\cline{2-5}
                                                                               & \cellcolor[HTML]{F79646}Running times & \cellcolor[HTML]{F79646}Compute performance ($s^{-1}$) & \cellcolor[HTML]{F79646}Limited time(s) & \cellcolor[HTML]{F79646}n \\ \hline
\multicolumn{1}{|c|}{\cellcolor[HTML]{0070C0}{\color[HTML]{FFFFFF} \textbf{(a)}}} & $n^2$                                 & $10^{10}$                                              & 3,600                                   & $6\times 10^6$            \\ \hline
\multicolumn{1}{|c|}{\cellcolor[HTML]{0070C0}{\color[HTML]{FFFFFF} \textbf{(b)}}} & $n^3$                                 & $10^{10}$                                              & 3,600                                   & 33,019                    \\ \hline
\multicolumn{1}{|c|}{\cellcolor[HTML]{0070C0}{\color[HTML]{FFFFFF} \textbf{(c)}}} & $100n^2$                              & $10^{10}$                                              & 3,600                                   & $6\times 10^5$            \\ \hline
\multicolumn{1}{|c|}{\cellcolor[HTML]{0070C0}{\color[HTML]{FFFFFF} \textbf{(d)}}} & $n\textrm{log} n$                     & $10^{10}$                                              & 3,600                                   & 906,316,482,853           \\ \hline
\multicolumn{1}{|c|}{\cellcolor[HTML]{0070C0}{\color[HTML]{FFFFFF} \textbf{(e)}}} & $2^n$                                 & $10^{10}$                                              & 3,600                                   & 45                        \\ \hline
\multicolumn{1}{|c|}{\cellcolor[HTML]{0070C0}{\color[HTML]{FFFFFF} \textbf{(f)}}} & $2^{2^{n}}$                           & $10^{10}$                                              & 3,600                                   & 5                         \\ \hline
\end{tabular}
}
\end{table}

}

% !Mode:: "TeX:UTF-8"

\chapter{}
\textbf{
Take the following list of functions and arrange them in ascending order of growth rate. That is, if function $g(n)$ immediately follows function $f(n)$ in your list, then it should be the case that $f(n)$ is $O(g(n))$.
}
\begin{enumerate}
  \item $f_1(n)=n^{2.5}$
  \item $f_2(n)=\sqrt{2n}$
  \item $f_3(n)=n+10$
  \item $f_4(n)=10^{n}$
  \item $f_5(n)=100^n$
  \item $f_6(n)=n^{2}\textrm{log} n$
  \item $g_1(n)=2^{\sqrt{\textrm{log} n}}$
  \item $g_2(n)=2^n$
  \item $g_4(n)=n^{4/3}$
  \item $g_3(n)=n(\textrm{log} n)^3$
  \item $g_5(n)=n^{\textrm{log} n}$
  \item $g_6(n)=2^{2^n}$
  \item $g_7(n)=2^{n^2}$
\end{enumerate}

\hspace*{\fill} \\

{
    It is easy to get the following basic sequence showed in Table~\ref{basic_order}.

\begin{table}[H]
\caption{Part of the ranking}
\label{basic_order}
\centering
\scalebox{1}{
\begin{tabular}{|c|c|l|l|}
\hline
2  & \multicolumn{3}{c|}{$f_2(n)=\sqrt{2n}$} \\ \hline
3  & \multicolumn{3}{c|}{$f_3(n)=n+10$}      \\ \hline
9  & \multicolumn{3}{c|}{$g_4(n)=n^{4/3}$}   \\ \hline
1  & \multicolumn{3}{c|}{$f_1(n)=n^{2.5}$}   \\ \hline
8  & \multicolumn{3}{c|}{$g_2(n)=2^{n}$}     \\ \hline
4  & \multicolumn{3}{c|}{$f_4(n)=10^{n}$}    \\ \hline
5  & \multicolumn{3}{c|}{$f_5(n)=100^{n}$}   \\ \hline
13 & \multicolumn{3}{c|}{$g_7(n)=2^{n^2}$}   \\ \hline
12 & \multicolumn{3}{c|}{$g_6(n)=2^{2^n}$}   \\ \hline
\end{tabular}
}
\end{table}

Then we compare the others with some pivotal functions in Table~\ref{basic_order}.

Consider $f_6(n)=n^2\textrm{log} n$, we can easily know that $f_6(n) = O(f_1(n))$ and $g_4(n)=O(f_6(n))$.

Maybe, it's difficult to find the position of $g_1(n)$. We compare $g_1(n)$ with $f_2(n)$. Let $m=\sqrt{\textrm{log} n}$, then we have $n=2^{m^2}$. As show in Equation~\ref{equ_1}. So we have $g_1(n)=O(f_2(n))$.
\begin{equation}\label{equ_1}
%\begin{align}
  \lim\limits_{n \to \infty} \frac{g_1(n)}{f_2(n)}  = \lim\limits_{m \to \infty} \frac{2^m}{\sqrt{2}\cdot 2^{\frac{1}{2}m^2}} = 0
%\end{align}
\end{equation}

It is obviously that $f_3(n)=O(g_3(n))$. Now we compare $g_3(n)=n(\textrm{log} n)^3$ with $g_4(n)=n^{4/3}$. Let $m=\textrm{log} n$, then $n=2^m$. And we can get Equation~\ref{equ_2}.
\begin{equation}\label{equ_2}
  \lim\limits_{n \to \infty} \frac{g_3(n)}{g_4(n)}  = \lim\limits_{m \to \infty} \frac{m^3}{2^{\frac{1}{3}m}} = 0
\end{equation}
Therefore, $g_3(n)=O(g_4(n))$.

Finally, we assure where $g_5(n)=n^{\textrm{log} n}$ should be placed. It is easy to know that $f_1(n)=O(g_5(n))$. Owing to the fact that $g_5(n)=n^{\textrm{log} n}=2^{\textrm{log} (n^{\textrm{log} n})}=2^{{\textrm{(log n)}^2}}$, we guess $g_5(n)$ is between $f_1(n)$ and $g_2(n)$. Let $m=\textrm{log} n$, then $n=2^m$. And we can get Equation~\ref{equ_3}.
\begin{equation}\label{equ_3}
  \lim\limits_{n \to \infty} \frac{g_5(n)}{g_2(n)}  = \lim\limits_{m \to \infty} \frac{2^{m^2}}{2^{2^m}} = 0
\end{equation}
So we have $g_5(n)=O(g_2(n))$. 

We summarize the final result in Table~\ref{final_order}.
\begin{table}[H]
\caption{Rank of functions}
\label{final_order}
\centering
\scalebox{1}{
\begin{tabular}{|c|c|l|l|}
\hline
7  & \multicolumn{3}{c|}{$g_1(n)=2^{\sqrt{\textrm{log}   n}}$} \\ \hline
2  & \multicolumn{3}{c|}{$f_2(n)=\sqrt{2n}$}                   \\ \hline
3  & \multicolumn{3}{c|}{$f_3(n)=n+10$}                        \\ \hline
10 & \multicolumn{3}{c|}{$g_3(n)=n(\textrm{log} n)^3$}         \\ \hline
9  & \multicolumn{3}{c|}{$g_4(n)=n^{4/3}$}                     \\ \hline
6  & \multicolumn{3}{c|}{$f_6(n)=n^2\textrm{log} n$}           \\ \hline
1  & \multicolumn{3}{c|}{$f_1(n)=n^{2.5}$}                     \\ \hline
11 & \multicolumn{3}{c|}{$g_5(n)=n^{\textrm{log} n}$}          \\ \hline
8  & \multicolumn{3}{c|}{$g_2(n)=2^{n}$}                       \\ \hline
4  & \multicolumn{3}{c|}{$f_4(n)=10^{n}$}                      \\ \hline
5  & \multicolumn{3}{c|}{$f_5(n)=100^{n}$}                     \\ \hline
13 & \multicolumn{3}{c|}{$g_7(n)=2^{n^2}$}                     \\ \hline
12 & \multicolumn{3}{c|}{$g_6(n)=2^{2^n}$}                     \\ \hline
\end{tabular}
}
\end{table}

}

% !Mode:: "TeX:UTF-8"

\chapter{}
\textbf{
Suppose now that you're given $n\times n$ grid graph $G$.(An $n\times n$ grid graph is just the adjacency graph of an $n\times n$ chessboard. To be completely precise, it is a graph whose node set is the set of all ordered pairs of natural numbers $(i,j)$, where $1\leq i \leq n$ and $1\leq j \leq n$; the nodes $(i,j)$ and $(k,l)$ are joined by an edge if and only if $|i-k|+|j-l|=1$.)
}

\textbf{
We use some of the terminology of the previous question. Again, each node $v$ is labeled by a real number $x_v$; you may assume that all these labels are distinct. Show how to find a local minimum of $G$ using only $O(n)$ probes to the nodes of $G$.(Note that $G$ has $n^2$ nodes.)
}
\hspace*{\fill} \\

We first give a example. Consider the grid graph show in figure~\ref{question_4_example}.
\\
\begin{figure}[h]
\centering
\includegraphics[width=0.4\textwidth]{figures/5.eps}
\caption{A grid graph example}\label{question_4_example}
\end{figure}

We can use divide-and-conquer to handle this problem. Intuitively, we can find the minimum element $tmin$ in the middle column and test whether both the left and right element are greater than $tmin$. If so, we can return $tmin$ directly, else we choose the one that less than $tmin$ and then recursive choose the minimum element.

Considering that we should declare the specific area, we can add a board to the grid graph $G$, and keep the elements on the boarders are always greater than the current element. We show the steps as follows.
\begin{enumerate}
  \item First, we add four boarders to $G$ as is showed in figure~\ref{question_4_example_2}. And fill the border with the value greater than all value in $G$;
  \item We calculate the minimum value of elements on the borders and on the middle row and middle column. And we get the current minimum value is 3;
  \item We test the untested neighbors and find the element 1 above it is less than it. So we choose this element as get into the sub-window at the up-left corner.
  \item We recursively calculate the minimum value of elements on the borders and on the middle row and middle column. And we get element 2. And then we can choose the element 1 on the left on element 2 as the local minimum of $G$.
\end{enumerate}

\begin{figure}[!htbp]
\centering
\includegraphics[width=0.5\textwidth]{figures/6.eps}
\caption{A grid graph example}\label{question_4_example_2}
\end{figure}


\begin{figure}[!htbp]
\centering
\includegraphics[width=0.5\textwidth]{figures/7.eps}
\caption{A grid graph example}\label{question_4_example_3}
\end{figure}


We give the algorithm in Alg~\ref{alg_4_1}.
\begin{algorithm}[!htbp]
\caption{Find local minimum value}
\label{alg_4_1}
\begin{algorithmic}[1]
\REQUIRE Grid $M$
\STATE Add four borders to $M$ and fill it with the value greater than the maximum value in $M$.
\STATE{Set the current windows size $S$ as Size($M$)+2}
\WHILE{$S > 0$}
    \STATE{Find the minimum element $elem$ in current four borders, middle row and middle column}
    \IF{$elem$ is the local minimum}
        \STATE{Return $elem$}
    \ELSE
        \STATE{Choose the windows which contains the neighbor $elemNB$ of $elme$ satisfying $elemNB<elem$}
        \STATE{Set $S$ as $\lfloor S/2\rfloor$ }
    \ENDIF
\ENDWHILE
\end{algorithmic}
\end{algorithm}
We can find that calculating the minimum value in each recursive step costs $O(n)$. And we have $T(n)=T(\frac{n}{4})+O(n)$. Therefore the complexity of running time is $O(n)$. 
% !Mode:: "TeX:UTF-8"

\chapter{}
\textbf{
5.Let $G=(V,E)$ be a directed graph, and suppose that for each node $v$, the number of edges into $v$ is equal to the number of edges out of $v$. That is for all $v$,
$$|\{(u,v):(u,v)\in E\}|=|\{(v,w):(v,w)\in E\}|$$
}

\textbf{
Let $x,y$ be two nodes of $G$, and suppose that there exist $k$ mutually edge-disjoint paths from $y$ to $x$? Give a proof or a counterexample with explanation.
}

It is true.
Consider the $k$ paths from $x$ to $y$. We can see that these $k$ path generate a flow. Its source $x$'s outer degree and sink $y$'s inner degree are both $k$, the other vertexes' outer degrees are equal to the inner degrees. Because the condition says the number of edges into each vertex's is equal to the number of those out of the vertex, $y$ has to find vertex to make its outer degree equal to its inner degree, which will remedy the inner degree of $x$.

% !Mode:: "TeX:UTF-8"

\chapter{}
\textbf{
6.考虑到地震救灾场景,$n$个伤员需要被尽快总往医院。在这个地区有$k$ 所医院,这$n$个人中每个人需要被送到距离他们目前的地点半小时车程以内的医院(因此不同的人将对医院有不同的选择,依赖于他们当前所在的地方)。同时,人们不想由于太多的病人送来而使得任何一个医院超负荷。医护人员通过移动电话联系,他们想共同解决是否可以为每个受伤的人选择一所医院,这种选择方式要求医院负荷是均衡的,即每个医院至多接受$\lceil n/k\rceil$ 的人。给出一个多项式时间的算法,它以关于这些人所在位置的给定信息作为输入并且确定这是不可能的。
}

我们将伤员($w_i$)和医院($h_j$)表示成点,根据题目条件,加上源点$s$ 和汇点$t$可以构成Fig~\ref{q6_1}。
\begin{figure}[H]
  \centering
  \includegraphics[width=0.7\textwidth]{figures/4.eps}\\
  \caption{Instance}\label{q6_1}
\end{figure}
图中$C_{hi}$ $(1\leq i\leq k)$ 表示各个医院的最大容量。因此只要执行最大流算法,就可以计算出结果。最大流算法的时间复杂度是多项式时间的。
% !Mode:: "TeX:UTF-8"

\chapter{}
\textbf{
7.考虑郊野环境中的移动通信场景,给定$n$个基站的位置,它们由平面上的点$b_1,b_2,...,b_n$来指定,以及$n$个手机用户的位置,它们也指定为平面上的点$p_1,p_2,...,p_n$,最后,给定一个域参数$\Delta>0$。如果能以下述这样的方式把每个电话分配给一个基站,我们就说这组便携式电话是完全联通的。
}

\textbf{每个手机被分到不同的基站,且如果位于$p_i$的电话被分配到位于$b_j$的基站,那么在点$p_i$和$b_j$之间的直线距离至多是$\Delta$。}

\textbf{假设在点$p_i$的用户决定开车向东经过$z$距离,需要修改手机对基站的分配(可能要几次)以便保持完全连通,给出一个多项式时间算法。(假定在此期间其他手机固定。)如果可能,报告一个电话到基站的分配序列;如果不可能,报告使得完全连通性不可能在维持的一个点。算法运行时间在$O(n^3)$时间。}

首先以$n$个基站位置为圆心,半径$\Delta$到$b_j$运行线段所在直线的交点,并计算出各个交点到$p_i$的距离,进行排序。遍历所有交点,如果是和圆第一次相交,就是进入点,如果第二次相交就是离开点,离开点的后继还是离开点,那么改点就是一个不可维持点。

该过程对某一个$p_i$,执行时间是$O(nlog n)$.





%%%%%%%%%% 正文部分内容  %%%%%%%%%%

\clearpage
\end{CJK*}                                        % 结束中文字体使用
\end{document}                                    % 结束全文
