% !Mode:: "TeX:UTF-8"

\chapter{}
\textbf{
Given an $O(n lgk)$-time algorithm to merge $k$ sorted lists into one sorted list, where $n$ is the total number of elements in all the input lists.(\emph{Hint}: Use a min-heap for k-way merging).
}
\hspace*{\fill} \\

We can generalize the 2-array merge to $k$-array merge. It is obviously that we can set $k$ points and put each of them at the first elements of the $k$ arrays. Then we should compare the $k$ elements pointed by $k$ the pointers and move the pointer pointing to the smallest one to the next element. As to the comparison of the $k$ elements, we can use min-heap to store and sorted them. The algorithm is showed in Alg~\ref{kmerge}.

\begin{algorithm}
\caption{Merge $k$ sorted array algorithm}
\label{kmerge}
\begin{algorithmic}[1]
\REQUIRE $k$ sorted arrays $A[1],A[2],...,A[k]$
\STATE Extract the first element of $A[1],A[2],...,A[k]$ respectively to get array $B$.
\STATE{Build a min-heap $H$ using array $B$}
\WHILE{$H\neq \phi$}
    \STATE{Move the root $r$ element to the result list R}
    \IF{the array $A[j]$ contains $r$ is not empty}
        \STATE{Move the next element in $A[j]$ to the root}
        \STATE{Heapify the heap}
    \ELSE
        \STATE{Move the last leaf to the root}
        \STATE{Heapify the heap}
    \ENDIF
\ENDWHILE
\STATE Return R
\end{algorithmic}
\end{algorithm}

We can see that it costs $lg k$ time to get the minimal element in the current min-heap and the total number of elements is $n$. So the time of this merging is $O(n lg k)$.