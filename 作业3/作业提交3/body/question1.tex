% !Mode:: "TeX:UTF-8"

\chapter{}
\textbf{
Given asymptotic upper and lower bounds for $T(n)$ in each of the following recurrences. Assume that $T(n)$ is constant for $n\leq 2$. Make your bounds as tight as possible, and justify your answers.
\begin{enumerate}
  \item $T(n)=2T(n/2)+n^4$.
  \item $T(n)=T(7n/10)+n$.
  \item $T(n)=16T(n/4)+n^2$.
  \item $T(n)=7T(n/3)+n^2$.
  \item $T(n)=7T(n/2)+n^2$.
  \item $T(n)=2T(n/4)+\sqrt{n}$.
  \item $T(n)=T(n-2)+n^2$.
\end{enumerate}
}

\hspace*{\fill} \\

1. According to master theory, $a=2$, $b=2$, $f(n)=n^4$. We have $f(n)/n^{log_ba}=n^4/n=n^3$. $af(n/b)/f(n)=2f(n/2)/f(n)<1$. It is case 3. So $T(n)=\Theta(n^4)$.

2. According to master theory, $a=1$, $b=10/7$, $f(n)=n$. We have $f(n)/n^{log_ba}=n/1=n$. $af(n/b)/f(n)=7/10<1$. It is case 3. So $T(n)=\Theta(n)$.

3. According to master theory, $a=16$, $b=4$, $f(n)=n^2$. We have $f(n)/n^{log_ba}=1=\Theta(lg^0n)$. It is case 2. So $T(n)=\Theta(lgn)$.

4. According to master theory, $a=7$, $b=3$, $f(n)=n^2$. We have $f(n)/n^{log_ba}=n^{0.23}=$. $af(n/b)/f(n)=7/9<1$. It is case 3. So $T(n)=\Theta(n^2)$.

5. According to master theory, $a=7$, $b=2$, $f(n)=n^2$. We have $f(n)/n^{log_ba}=n^{-0.81}=$. It is case 1. So $T(n)=\Theta(n^{log_27})$.

6. According to master theory, $a=2$, $b=4$, $f(n)=n^\frac{1}{2}$. We have $f(n)/n^{log_ba}=1=\Theta(lg^0n)$. It is case 2. So $T(n)=\Theta(lgn)$.

7.
\begin{align*}
 T(n) &= n^2 + T(n-2) \\
      &= n^2 + (n-2)^2 + T(n-4)\\
      &...\\
      &= n^2 + (n-2)^2 + ... + 4^2 + 2^2 \textrm{或} = n^2 + (n-2)^2 + ... + 3^2 + 1^2
\end{align*}
We can get that following relations:
\begin{align*}
 T(n) &\geq 1^2+3^3+...+(2\lceil\frac{n}{2}\rceil-1)^2 \\
      &> \frac{1}{2}(1^2+2^2+3^2+...+(2\lceil\frac{n}{2}\rceil-1)^2) \\
      &> \frac{1}{2}(1^2+2^2+3^2+...+(n-1)^2) \\
      &=\frac{n(n-1)(2n-1)}{12} \\
      &=\Theta(n^3)
\end{align*}
and
\begin{align*}
 T(n) &\leq 2^2+4^3+...+(2\lceil\frac{n}{2}\rceil)^2 \\
      &< \frac{1}{2}(2^2+3^2+...+(2\lceil\frac{n}{2}\rceil+1)^2) \\
      &< \frac{1}{2}(1^2+2^2+3^2+...+(2\frac{n+1}{2}+1)^2) \\
      &= \frac{1}{2}(1^2+2^2+3^2+...+(n+2)^2) \\
      &< \frac{(n+2)(n+3)(2n+5)}{12} \\
      &=\Theta(n^3)
\end{align*}
Therefore, we can conclude that $T(n)=\Theta(n^3)$
