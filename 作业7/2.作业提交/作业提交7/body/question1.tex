% !Mode:: "TeX:UTF-8"

\chapter{}
\textbf{
1.为了评价在一个有向图中两个结点是``连通得有多好",人们不仅可以看它们之间的最短路径的长度,而且也可以计数最短路径的条数。
}

\textbf{
在边的费用具有某些限制的条件,已经证明了这是一个可以有效求解的问题。假设给我们一个边上带有费用的有向图$G=(V,E)$;费用可能是正的或者负的,但是图中的每个圈严格有着正的费用。还给定两个结点$v,w\in V$。给出一个有效的算法计算$G$中最短路$v-w$路径的条数。(算法不必列出所有的路径;只要数目就足够了。)
}

\hspace*{\fill} \\

可以采用Floyd算法计算每对结点的最短路径,每次更新时判断更新路径$d(v,t)+d(t,w)$与当前路径$d(v,w)$的大小。如果大$d(v,t)+d(t,w)<d(t,w)$,则设置$C(v,w)=1$,如果$d(v,t)+d(t,w)=d(t,w)$,则将$C(v,w)$增加1。算法设计如下~\ref{alg_1}。
\begin{algorithm}
\caption{Optimal path count}
\label{alg_1}
\begin{algorithmic}[1]
\REQUIRE graph $G(V,E)$, $v$, $w$
\STATE{Initialize matrix $dist$ as direct path cost}
\STATE{Initialize matrix $C$ as 0}
\FOR{$k$ from $1$ to $|V|$}
    \FOR{$i$ from $1$ to $|V|$}
        \FOR{$j$ from $1$ to $|V|$}
            \IF{$dist[i][k]+dist[k][j]<dist[i][j]$}
                \STATE{$dist[i][j]=dist[i][k]+dist[k][j]$}
                \STATE{$C[i][j]=1$}
            \ELSIF{$dist[i][k]+dist[k][j]=dist[i][j]$}
                \STATE{$C[i][j]+=1$}
            \ENDIF
        \ENDFOR
    \ENDFOR
\ENDFOR
\STATE{Return $C[v][2]$}
\end{algorithmic}
\end{algorithm}
