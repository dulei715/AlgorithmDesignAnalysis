% !Mode:: "TeX:UTF-8"

\chapter{}
\textbf{
2.某加油站有一个大的地下储油罐存储汽油;这个罐一次至多存$L$ 加仑。订购油是相当贵的,因此他们希望订货要比较少。每次订货,他们除了所订购油的费用之外,还需要付固定价格$P$的运费。但是,每加仑油多存1天的费用是$c$,因此提前订购会增加存储的费用。
}

\textbf{
他们计划在冬天休息一周,希望储罐到休业的时间是空的。幸运的是,基于多年的经验,他们对于直到这个时间之前的每一天将需要多少油有着精确的规划。假定到他们休业还有$n$天,对于$i=1,2,...,n$的每一天$i$他们需要$g_i$加仑汽油。假定储罐在第0天结束时是空的。给出一个算法来决定他们应该在哪天订货,以及订多少,以使得他们的总费用最小。
}


令$W(i,j)$表示在第$i$天加油,第$j$天刚好用完的最小消耗(中间可以加油)。于是我们可以得到:
\begin{equation}
W(i,j)=
\left\{
             \begin{array}{lr}
             P,
             & (i=j) \\
             \min\{\min\limits_{k=i}^{j-1}\{W(i,k)+W(k+1,j)\}, P+\sum\limits_{m=i+1}^{j}(m-i)cg_m\},
             & (i<j)
             \end{array}
\right.
\end{equation}
算法设计如下~\ref{alg_2}
\begin{algorithm}
\caption{Optimal path count}
\label{alg_2}
\begin{algorithmic}[1]
\REQUIRE 4
\STATE{Initialize matrix $G$ and $W$}
\FOR{$i$ from 1 to $n$}
    \FOR{$j$ from $i+1$ to $n$}
        \STATE{Set $G[i][j]=\sum\limits_{m=i+1}^{j}(m-i)cg_m$}
    \ENDFOR
\ENDFOR
\FOR{$i$ from 1 to $n$}
    \FOR{$j$ from $i$ to $n$}
        \STATE{Set $W[i][j][1]=\min\{\min\limits_{k=i}^{j-1}\{W[i][k]+W[k+1][j]\}, P+G[i][j]\}$}
        \STATE{Set $W[i][j][2]$ to be its deriving}
    \ENDFOR
\ENDFOR

\STATE{Backtrack from $W[1][n]$ by $W[i][j][2]$}
\STATE{The backtracking result}
\end{algorithmic}
\end{algorithm}

