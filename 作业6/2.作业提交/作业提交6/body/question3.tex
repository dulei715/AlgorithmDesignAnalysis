% !Mode:: "TeX:UTF-8"

\chapter{}

\textbf{
Suppose it's nearing the end of the semester and you're taking \emph{n} courses, each with a final project that still has to be done. Each project will be graded on the following scale: It will be assigned an integer number on a scale of 1 to $g>1$, higher numbers being better grades. Your goal, of course, is to maximize your average grade on the \emph{n} projects.
}

\textbf{
You have a total of $H>n$ hours in which to work on the $n$ projects cumulatively, and you want to decide how to divide up this time. For simplicity, assume $H$ is a positive integer, and you'll spend an integer number of hours on each project. To figure out how best divide up your time, you've come up with a set of functions $f_i: i=1,2,...,n$(rough estimates, of course) for each of your $n$ courses; if you spend $h\leq H$ hours on the project for course $i$, you'll get a grade of $f_i(h)$.(You may assume that the functions $f_i$ are \emph{$f_i$} are \emph{nondecreasing} if $h<h'$, then $f_i(h)\leq f_i(h')$.)
}

\textbf{
So the problem is: Given these functions{$f_i$}, decide how many hours to spend on each project(in integer values only) so that your average grade, as computed according to the $f_i$, is as large as possible. In order to be efficient, the running time of your algorithm should be polynomial in $n,g$, and $H$; none of these quantities should appear as an exponent in your running time.
}

\hspace*{\fill} \\
Let $F(i,h)$ be the total grades satisfying that there are $i$ courses and there are $h$ hours.
We can easy to get the following equation:
\begin{equation}
F(i,h)=
\left\{
             \begin{array}{lr}
             f_1(h),
             & i=1 \\
             \max\limits_{k=1}^{h-i+1}\{F(i-1,h-k)+f_i(k)\},
             & i\geq 2
             \end{array}
\right.
\end{equation}

Then we have the algorithm showed as~\ref{alg_3}
\begin{algorithm}
\caption{Maximum grades}
\label{alg_3}
\begin{algorithmic}[1]
\REQUIRE the quantity of courses $n$; a series $f_1,f_2,....f_n$ of $n$ courses; total hours $h$
\STATE{Initialize a upper triangular matrix $[F_{i,h}]_{i<h}$}
\FOR{$h$ from $1$ to $H$}
    \STATE{$F[1][h]=f_1(h)$}
\ENDFOR
\FOR{$i$ from $2$ to $n$}
    \FOR{$h$ from $1$ to $H$}
        \STATE{Set $F(i,h)$ as the maximum value of $F(i-1,h-k)+f_i(k)$ for $k=1$ to $h-i+1$}
    \ENDFOR
\ENDFOR
\STATE{Return $F(n,H)$}
\end{algorithmic}
\end{algorithm}

The time costs is roughly $O(H^3)$
