% !Mode:: "TeX:UTF-8"
%\setCJKfamilyfont{hwhp}{华文琥珀}
%\newcommand{\hwhp}{\CJKfamily{hwhp}}
%\newfontfamily\tempus{Tempus Sans ITC}

%%%%%%%%%% Fonts Definition and Basics %%%%%%%%%%%%%%%%%
\newcommand{\song}{\CJKfamily{song}}    % 宋体
\newcommand{\fs}{\CJKfamily{fs}}        % 仿宋体
\newcommand{\kai}{\CJKfamily{kai}}      % 楷体
\newcommand{\hei}{\CJKfamily{hei}}      % 黑体
\newcommand{\li}{\CJKfamily{li}}        % 隶书

%\setCJKfamilyfont{hwhp}{华文琥珀}
%\newcommand{\hwhp}{\CJKfamily{hwhp}}

%\newfontfamily\tempus{Tempus Sans ITC}

\newcommand{\yihao}{\fontsize{26pt}{26pt}\selectfont}       % 一号, 1.倍行距
\newcommand{\xiaoyi}{\fontsize{24pt}{24pt}\selectfont}      % 小一, 1.倍行距
\newcommand{\erhao}{\fontsize{22pt}{22pt}\selectfont}       % 二号, 1.倍行距
\newcommand{\xiaoer}{\fontsize{18pt}{18pt}\selectfont}      % 小二, 单倍行距
\newcommand{\sanhao}{\fontsize{16pt}{16pt}\selectfont}      % 三号, 1.倍行距
\newcommand{\xiaosan}{\fontsize{15pt}{15pt}\selectfont}     % 小三, 1.倍行距
\newcommand{\sihao}{\fontsize{14pt}{14pt}\selectfont}       % 四号, 1.0倍行距
\newcommand{\xiaosi}{\fontsize{12.5pt}{12.5pt}\selectfont}      % 小四, 1.倍行距
\newcommand{\wuhao}{\fontsize{10.5pt}{10.5pt}\selectfont}   % 五号, 单倍行距
\newcommand{\xiaowu}{\fontsize{9pt}{9pt}\selectfont}        % 小五, 单倍行距
\setlength{\headheight}{20pt}
%\CJKcaption{gb_452}
\CJKtilde  % 重新定义了波浪符~的意义
\newcommand\prechaptername{Question }
\newcommand\postchaptername{}

% 调整罗列环境的布局
\setitemize{leftmargin=3em,itemsep=0em,partopsep=0em,parsep=0em,topsep=-0em}
\setenumerate{leftmargin=3em,itemsep=0em,partopsep=0em,parsep=0em,topsep=0em}


%避免宏包 hyperref 和 arydshln 不兼容带来的目录链接失效的问题。
\def\temp{\relax}
\let\temp\addcontentsline
\gdef\addcontentsline{\phantomsection\temp}

% 自定义项目列表标签及格式 \begin{publist} 列表项 \end{publist}
\newcounter{pubctr} %自定义新计数器
\newenvironment{publist}{%%%%%定义新环境
\begin{list}{[\arabic{pubctr}]} %%标签格式
    {
     \usecounter{pubctr}
     \setlength{\leftmargin}{2em}     % 左边界 \leftmargin =\itemindent + \labelwidth + \labelsep
     \setlength{\itemindent}{0em}     % 标号缩进量
     \setlength{\labelsep}{1em}       % 标号和列表项之间的距离,默认0.5em
     \setlength{\rightmargin}{0em}    % 右边界
     \setlength{\topsep}{0ex}         % 列表到上下文的垂直距离
     \setlength{\parsep}{0ex}         % 段落间距
     \setlength{\itemsep}{0ex}        % 标签间距
     \setlength{\listparindent}{0pt} % 段落缩进量
    }}
{\end{list}}%%%%%


\makeatletter
\renewcommand\normalsize{
  \@setfontsize\normalsize{12.5pt}{12.5pt} % 小四对应12pt
  \setlength\abovedisplayskip{4pt}
  \setlength\abovedisplayshortskip{4pt}
  \setlength\belowdisplayskip{\abovedisplayskip}
  \setlength\belowdisplayshortskip{\abovedisplayshortskip}
    \let\@listi\@listI}
\def\defaultfont{\renewcommand{\baselinestretch}{1.65}\normalsize\selectfont}


% 设置行距和段落间垂直距离

\setlength{\baselineskip}{20pt}
\renewcommand{\CJKglue}{\hskip 0.5pt plus \baselineskip} % 加大字间距,使每行35个字


\makeatother

%%%%%%%%%%%%% Contents %%%%%%%%%%%%%%%%%
\renewcommand{\contentsname}{目\qquad 录}
\setcounter{tocdepth}{2}
\titlecontents{chapter}[0em]{\xiaosi\hei}%
             {\prechaptername~~\thecontentslabel~~\postchaptername~~~}{} %
             {\titlerule*[5pt]{$\cdot$}\xiaosi\contentspage}
\titlecontents{section}[2em]{\xiaosi\song} %
            {\thecontentslabel\quad}{} %
            {\hspace{.25em}\titlerule*[5pt]{$\cdot$}\xiaosi\contentspage}
\titlecontents{subsection}[4em]{\xiaosi\song} %
            {\thecontentslabel\quad}{} %
            {\hspace{.25em}\titlerule*[5pt]{$\cdot$}\xiaosi\contentspage}
\renewcommand{\cftdotsep}{1.1}
\renewcommand{\listfigurename}{插图索引}
\setcounter{lofdepth}{1}
%\titlefigures{chapter}[1em]{\xiaosi\hei}%
             %{\prechaptername~~\thecontentslabel~~\postchaptername~~~}{} %
            % {\titlerule*[10pt]{$\cdot$}\xiaosi\contentspage}
\renewcommand{\listtablename}{附表索引}


%%删除表格和插图因章不同中的空行%%%
\makeatletter
\def\@chapter[#1]#2{\ifnum \c@secnumdepth >\m@ne
                       \if@mainmatter
                         \refstepcounter{chapter}%
                         \typeout{\@chapapp\space\thechapter.}%
                         \addcontentsline{toc}{chapter}%
                                   {\protect\numberline{\thechapter}#1}%
                       \else
                         \addcontentsline{toc}{chapter}{#1}%
                       \fi
                    \else
                      \addcontentsline{toc}{chapter}{#1}%
                    \fi
                    \chaptermark{#1}%
                    \if@twocolumn
                      \@topnewpage[\@makechapterhead{#2}]%
                    \else
                      \@makechapterhead{#2}%
                      \@afterheading
                    \fi}
\makeatother


%%%%%%%%%% Chapter and Section %%%%%%%%%%%%%%%%%
\setcounter{secnumdepth}{4}
\setlength{\parindent}{2em}
\renewcommand{\chaptername}{\prechaptername\arabic{chapter}\postchaptername}
\titleformat{\chapter}{\centering\xiaoer\hei}{\chaptername}{1em}{}
\titlespacing{\chapter}{0pt}{0pt}{18pt}
\titleformat{\section}{\xiaosan\hei}{\thesection}{1em}{}
\titlespacing{\section}{0pt}{12pt}{12pt}
\titleformat{\subsection}{\sihao\hei}{\thesubsection}{0.5em}{}
\titlespacing{\subsection}{0pt}{6pt}{6pt}
\titleformat{\subsubsection}{\xiaosi\hei}{\thesubsubsection}{0.5em}{}
\titlespacing{\subsubsection}{0pt}{6pt}{6pt}

%%%%%%%%%% Table, Figure and Equation %%%%%%%%%%%%%%%%%
\renewcommand{\tablename}{Table} % 插表题头
\renewcommand{\figurename}{Figure} % 插图题头
\renewcommand{\thefigure}{\arabic{chapter}.\arabic{figure}} % 使图编号为 7.1 的格式 %\protect{~}
\renewcommand{\thetable}{\arabic{chapter}.\arabic{table}}% 使表编号为 7.1 的格式
\renewcommand{\theequation}{\arabic{chapter}.\arabic{equation}}% 使公式编号为 7-1 的格式
\renewcommand{\thesubfigure}{(\alph{subfigure})}%使子图编号为 (a)的格式
\renewcommand{\thesubtable}{(\alph{subtable})} %使子表编号为 (a)的格式
\makeatletter
\renewcommand{\p@subfigure}{\thefigure~} %使子图引用为 7-1 a) 的格式,母图编号和子图编号之间用~加一个空格
\makeatother


%% 定制浮动图形和表格标题样式
\makeatletter
\long\def\@makecaption#1#2{%
   \vskip\abovecaptionskip
   \sbox\@tempboxa{\centering\wuhao\hei{#1~~#2} }%
   \ifdim \wd\@tempboxa >\hsize
     \centering\wuhao\hei{#1~~#2} \par
   \else
     \global \@minipagefalse
     \hb@xt@\hsize{\hfil\box\@tempboxa\hfil}%
   \fi
   \vskip\belowcaptionskip}
\makeatother
\captiondelim{~~~~} %用来控制longtable表头分隔符

%%%%%%%%%% Theorem Environment %%%%%%%%%%%%%%%%%
\theoremstyle{plain}
\theorembodyfont{\song\rmfamily}
\theoremheaderfont{\hei\rmfamily}
\newtheorem{theorem}{定理~}[chapter]
\newtheorem{lemma}{引理~}[chapter]
\newtheorem{axiom}{公理~}[chapter]
\newtheorem{proposition}{命题~}[chapter]
\newtheorem{corollary}{推论~}[chapter]
\newtheorem{definition}{定义~}[chapter]
\newtheorem{conjecture}{猜想~}[chapter]
\newtheorem{example}{例~}[chapter]
\newtheorem{remark}{注~}[chapter]
%\floatname{algorithm}{算法}%将英文的algorithm改为算法
\renewcommand{\algorithmicrequire}{\textbf{Input:}}
\renewcommand{\algorithmicensure}{\textbf{Output:}}
\newenvironment{proof}{\noindent{\hei 证明:}}{\hfill $ \square $ \vskip 4mm}
\theoremsymbol{$\square$}

%%%%%%%%%% Page: number, header and footer  页码%%%%%%%%%%%%%%%%%

%\frontmatter 或 \pagenumbering{roman}
%\mainmatter 或 \pagenumbering{arabic}
\makeatletter
\renewcommand\frontmatter{\clearpage
  \@mainmatterfalse
  \pagenumbering{Roman}} % 正文前罗马字体编号
\makeatother


%%%%%%%%%% References %%%%%%%%%%%%%%%%%
\renewcommand{\bibname}{参考文献}
% 重定义参考文献样式,来自thu
\makeatletter
\renewenvironment{thebibliography}[1]{%
   \chapter*{\bibname}%
   \xiaosi
   \list{\@biblabel{\@arabic\c@enumiv}}%
        {\renewcommand{\makelabel}[1]{##1\hfill}
         \setlength{\baselineskip}{21pt}
         \settowidth\labelwidth{0.5cm}
         \setlength{\labelsep}{0pt}
         \setlength{\itemindent}{0pt}
         \setlength{\leftmargin}{\labelwidth+\labelsep}
         \addtolength{\itemsep}{-0.7em}
         \usecounter{enumiv}%
         \let\p@enumiv\@empty
         \renewcommand\theenumiv{\@arabic\c@enumiv}}%
    \sloppy\frenchspacing
    \clubpenalty4000%
    \@clubpenalty \clubpenalty
    \widowpenalty4000%
    \interlinepenalty4000%
    \sfcode`\.\@m}
   {\def\@noitemerr
     {\@latex@warning{Empty `thebibliography' environment}}%
    \endlist\frenchspacing}
\makeatother

\addtolength{\bibsep}{5pt} % 增加参考文献间的垂直间距
\setlength{\bibhang}{2em} %每个条目自第二行起缩进的距离

% 参考文献引用作为上标出现
\newcommand{\mycite}[1]{\scalebox{1.3}[1.3]{\raisebox{-0.65ex}{\cite{#1}}}}

%% 引用格式
\bibpunct{[}{]}{,}{s}{}{,}

%%%%%%%%%% Cover %%%%%%%%%%%%%%%%%
% 封面、摘要、版权、致谢格式定义
\makeatletter

%\def\dtitle#1{\def\@dtitle{#1}}\def\@dtitle{}
\def\ctitle#1{\def\@ctitle{#1}}\def\@ctitle{}
\def\etitle#1{\def\@etitle{#1}}\def\@etitle{}
\def\caffil#1{\def\@caffil{#1}}\def\@caffil{}
\def\cmacrosubject#1{\def\@cmacrosubject{#1}}\def\@cmacrosubject{}
\def\cmacrosubjecttitle#1{\def\@cmacrosubjecttitle{#1}}\def\@cmacrosubjecttitle{}
\def\csubject#1{\def\@csubject{#1}}\def\@csubject{}
\def\csubjecttitle#1{\def\@csubjecttitle{#1}}\def\@csubjecttitle{}
\def\cmajor#1{\def\@cmajor{#1}}\def\@cmajor{}
\def\cauthor#1{\def\@cauthor{#1}}\def\@cauthor{}
\def\cauthortitle#1{\def\@cauthortitle{#1}}\def\@cauthortitle{}
\def\csupervisor#1{\def\@csupervisor{#1}}\def\@csupervisor{}
\def\csupervisortitle#1{\def\@csupervisortitle{#1}}\def\@csupervisortitle{}
\def\cdate#1{\def\@cdate{#1}}\def\@cdate{}
\def\untitle#1{\def\@untitle{#1}}\def\@untitle{}
\def\declaretitle#1{\def\@declaretitle{#1}}\def\@declaretitle{}
\def\declarecontent#1{\def\@declarecontent{#1}}\def\@declarecontent{}
\def\authorizationtitle#1{\def\@authorizationtitle{#1}}\def\@authorizationtitle{}
\def\authorizationcontent#1{\def\@authorizationcontent{#1}}\def\@authorizationconent{}
\def\authorizationadd#1{\def\@authorizationadd{#1}}\def\@authorizationadd{}
\def\authorsigncap#1{\def\@authorsigncap{#1}}\def\@authorsigncap{}
\def\supervisorsigncap#1{\def\@supervisorsigncap{#1}}\def\@supervisorsigncap{}
\def\signdatecap#1{\def\@signdatecap{#1}}\def\@signdatecap{}
\long\def\cabstract#1{\long\def\@cabstract{#1}}\long\def\@cabstract{}
\long\def\eabstract#1{\long\def\@eabstract{#1}}\long\def\@eabstract{}
\def\ckeywords#1{\def\@ckeywords{#1}}\def\@ckeywords{}
\def\ekeywords#1{\def\@ekeywords{#1}}\def\@ekeywords{}
\def\cheading#1{\def\@cheading{#1}}\def\@cheading{}
\def\cnumber#1{\def\@cnumber{#1}}\def\@cnumber{}
\def\csecret#1{\def\@csecret{#1}}\def\@csecret{}
\def\chnunumer#1{\def\@chnunumer{#1}}\def\@chnunumer{}
\def\cclassnumber#1{\def\@cclassnumber{#1}}\def\@cclassnumber{}
\def\chnuname#1{\def\@chnuname{#1}}\def\@chnuname{}
\def\cchair#1{\def\@cchair{#1}}\def\@cchair{}
\def\ddate#1{\def\@ddate{#1}}\def\@ddate{}
%英文内封
\def\ename#1{\def\@ename{#1}}\def\@ename{}
\def\cbe#1{\def\@cbe{#1}}\def\@cbe{}
%\def\cms#1{\def\@cms{#1}}\def\@cms{}
\def\cdegree#1{\def\@cdegree{#1}}\def\@cdegree{}
\def\cclass#1{\def\@cclass{#1}}\def\@cclass{}
\def\emajor#1{\def\@emajor{#1}}\def\@emajor{}
\def\ehnu#1{\def\@ehnu{#1}}\def\@ehnu{}
\def\esupervisor#1{\def\@esupervisor{#1}}\def\@esupervisor{}
\def\edate#1{\def\@edate{#1}}\def\@edate{}
\def\elevel#1{\def\@elevel{#1}}\def\@elevel{}


\newlength{\@title@width}
\def\@put@covertitle#1{\makebox[\@title@width][s]{#1}}
% 定义封面
\def\makecover{
%\cleardoublepage%
   \phantomsection
    \pdfbookmark[-1]{\@ctitle}{ctitle}

    \begin{titlepage}
    \begin{center}

      \setlength{\@title@width}{3.5cm}
       {
  \begin{tabular}{lcclc}
   \xiaosi\hei{学校代号}&  \underline{\makebox[\@title@width][c]{\@chnunumer}}&\qquad \qquad \qquad \qquad \qquad & \xiaosi\hei{学\qquad 号}&  \underline{\makebox[\@title@width][c]{\@cnumber}}\\
   \xiaosi\hei{分~~类~~~号}&  \underline{\makebox[\@title@width][c]{\@cclassnumber}}&\qquad \qquad \qquad \qquad \qquad & \xiaosi\hei{密\qquad 级}&  \underline{\makebox[\@title@width][c]{\@csecret}} \\
   \end{tabular}
   }

 \begin{figure}[h]
  \centering
  \includegraphics[width=0.3\textwidth]{figures/Hnulogo}
  \end{figure}
      \vspace*{1cm}
      {\hei\erhao \@cheading}

      \vspace*{1cm}


      \begin{center}
      \begin{spacing}{1.5}
      \hei\yihao \@ctitle
      \end{spacing}
      \end{center}

      \vspace{\baselineskip}
      \setlength{\@title@width}{6.5cm}
  {

  \begin{spacing}{2.1}
   \xiaosi\hei{学位申请人姓名} \xiaosi\song\underline{\makebox[\@title@width][l]{\qquad\@cauthor}} \\
   \xiaosi\hei{培~~~~养~~~~~单~~~~~位} \xiaosi\song\underline{\makebox[\@title@width][l]{\qquad\@caffil}} \\
   \xiaosi\hei{导师姓名及职称} \xiaosi\song\underline{\makebox[\@title@width][l]{\qquad\@csupervisor}} \\
   \xiaosi\hei{学~~~~科~~~~~专~~~~~业} \xiaosi\song\underline{\makebox[\@title@width][l]{\qquad\@csubject}} \\
   \xiaosi\hei{研~~~~究~~~~~方~~~~~向} \xiaosi\song\underline{\makebox[\@title@width][l]{\qquad\@cmajor}}\\
   \xiaosi\hei{论~文~提~交~日~~期} \xiaosi\song\underline{\makebox[\@title@width][l]{\qquad\@cdate}} \\
  %\end{tabular}
  \end{spacing}
 }
 \end{center}


\clearpage
\thispagestyle{empty} %去掉页眉页脚

\noindent
\makebox[2.59cm][s]{}{\begin{tabular}{ll}
\xiaosi\hei 学校代号:\xiaosi\song~~\@chnunumer \\
\xiaosi\hei 学\qquad~号:\xiaosi\song~~\@cnumber\\
\xiaosi\hei 密\qquad~级:\xiaosi\song~~\@csecret\\
\end{tabular}
}

%
\vspace{5\baselineskip}

\noindent
\makebox[2.59cm][s]{}{
\xiaoer\song \@chnuname \@cheading
}
\\
\vspace{4\baselineskip}

\begin{spacing}{2}
\hangafter=1\hangindent=2.7cm   %换行后自动缩进
{\noindent
\makebox[2.59cm][s]{} {\hei\erhao\@ctitle}}
\end{spacing}
\vspace{4\baselineskip}

\setlength{\@title@width}{6.8cm}
  {
  \begin{spacing}{2}
  \xiaosi
   \noindent
  \makebox[2.59cm][s]{}{
  \begin{tabular}{lc}
   \underline{\xiaosi\hei学位申请人姓名:\song\makebox[\@title@width][l]{\qquad\qquad\@cauthor}} \\
   \underline{\xiaosi\hei导师姓名及职称:\song\makebox[\@title@width][l]{\qquad\qquad\@csupervisor}} \\
   \underline{\xiaosi\hei培~~~~养~~~~~单~~~~位:\song\makebox[\@title@width][l]{\qquad\qquad\@caffil}} \\
   \underline{\xiaosi\hei专~~~~业~~~~~名~~~~称:\song\makebox[\@title@width][l]{\qquad\qquad\@csubject}} \\
   \underline{\xiaosi\hei论~文~提~交~日~期:\song\makebox[\@title@width][l]{\qquad\qquad\@cdate}} \\
   \underline{\xiaosi\hei论~文~答~辩~日~期:\song\makebox[\@title@width][l]{\qquad\qquad\@ddate}}\\
   \underline{\xiaosi\hei答辩委员会主席:\song\makebox[\@title@width][l]{\qquad\qquad\@cchair}} \\
  \end{tabular}
  %\end{tabular}
    }
   \end{spacing}
    }

\clearpage
\thispagestyle{empty} %去掉页眉页脚

\begin{center}
\qquad\\
 \begin{spacing}{2.5}
 \xiaosan \@etitle

 \end{spacing}



 \begin{spacing}{2}
 \xiaosi
 by\\
 \@ename \\
 \@cbe\\
% \@cms\\
 A~\@cdegree~submitted in partial satisfaction of the\\
 requirements for the degree of\\
 \@cclass\\
 in\\
\@emajor\\
in the\\
Graduate school\\
 of\\
 \@ehnu\\

 \vspace{2\baselineskip}

Supervisor\\
\@elevel~~ \@esupervisor\\
\@edate
\end{spacing}

\end{center}


\end{titlepage}

%  另起一页: 独创性声明和学位论文版权使用授权书

\pagestyle{fancy}
\fancyhf{}
\fancyfoot[C]{\song\xiaowu ~\thepage~}
\renewcommand{\headrulewidth}{0pt}

    \addcontentsline{toc}{chapter}{学位论文原创性声明和学位论文版权使用授权书}{
    \setcounter{page}{1}
    \qquad\\
    \begin{center}\hei\xiaoer{\@untitle}\end{center}\par
    \begin{center}\hei\xiaoer{\@declaretitle}\end{center}\par
    \song\defaultfont{\@declarecontent}\par
    \vspace*{1cm}
    {\song\xiaosi
    \@authorsigncap \makebox[2.5cm][s]{}
    \@signdatecap \makebox[2cm][s]{} 年 \makebox[1cm][s]{} 月 \makebox[1cm][s]{} 日
    }
    \vspace{0.6\baselineskip}
    \begin{center}\hei\xiaoer{\@authorizationtitle}\end{center}\par
    {
    \vspace{1.2\baselineskip}
    \song\defaultfont{\@authorizationcontent}
    \begin{tabular}{ll}
     \song\defaultfont\@authorizationadd\par&\\
    &1、保密\song\xiaoer{$\Box$}\song\xiaosi ,在\underline{\qquad}年解密后适用于本授权书\\
    &2、不保密\song\xiaoer{$\Box$}。\\
    &(请在以上相应方框内打"$\surd$") \\
    \end{tabular}
    }
    \vspace{2\baselineskip}

    {
    \song\xiaosi
      \@authorsigncap \makebox[3.5cm][s]{}  \@signdatecap \makebox[1.5cm][s]{} 年 \makebox[1cm][s]{} 月 \makebox[1cm][s]{} 日 \\
      \indent
      \@supervisorsigncap \makebox[3.5cm][s]{}  \@signdatecap \makebox[1.5cm][s]{} 年 \makebox[1cm][s]{} 月 \makebox[1cm][s]{} 日
    }
    }


%%%%%%%%%%%%%%%%%%%   Abstract and Keywords  %%%%%%%%%%%%%%%%%%%%%%%
\clearpage

\pagestyle{fancy}
  \fancyhf{}
\fancyhead[CO]{\song\xiaowu \@cheading}
\fancyhead[CE]{\song\xiaowu \@ctitle}
\fancyfoot[C]{\song\xiaowu ~\thepage~}
\makeatletter %双线页眉
\def\headrule{{\if@fancyplain\let\headrulewidth\plainheadrulewidth\fi%
\hrule\@height 1.0pt \@width\headwidth\vskip1pt %上面线为1pt粗
\hrule\@height 0.5pt\@width\headwidth  %下面0.5pt粗
\vskip-2\headrulewidth\vskip-1pt}      %两条线的距离1pt
\vspace{7mm} %双线与下面正文之间的垂直间距
}

\fancypagestyle{plain}{% 设置开章页页眉页脚风格
    \fancyhf{}%
\fancyhead[CO]{\song\xiaowu \@cheading}
\fancyhead[CE]{\song\xiaowu \@ctitle}
\fancyfoot[C]{\song\xiaowu ~\thepage~} %首页页脚格式
%双线页眉的设置
\makeatletter %双线页眉
\def\headrule{{\if@fancyplain\let\headrulewidth\plainheadrulewidth\fi%
\hrule\@height 1.0pt \@width\headwidth\vskip1pt %上面线为1pt粗
\hrule\@height 0.5pt\@width\headwidth  %下面0.5pt粗
\vskip-2\headrulewidth\vskip-1pt}      %两条线的距离1pt
\vspace{7mm} %双线与下面正文之间的垂直间距
}

}
\addcontentsline{toc}{chapter}{摘~要}
\chapter*{\centering\xiaoer\ 摘\qquad 要}
\song\defaultfont
\@cabstract
\vspace{\baselineskip}

%\hangafter=1\hangindent=52.3pt\noindent   %如果取消该行注释,关键词换行时将会自动缩进
\noindent
{\hei\xiaosi 关键词: \@ckeywords}

%%%%%%%%%%%%%%%%%%%   English Abstract  %%%%%%%%%%%%%%%%%%%%%%%%%%%%%%
\clearpage

\addcontentsline{toc}{chapter}{Abstract}
\chapter*{\centering\xiaoer \bf{Abstract}}
%\vspace{\baselineskip}
\@eabstract
\vspace{\baselineskip}

%\hangafter=1\hangindent=60pt\noindent  %如果取消该行注释,KEY WORDS换行时将会自动缩进
\noindent
{\xiaosi\bf{Key Words: \@ekeywords}}
}
\clearpage
\makeatother
