% !Mode:: "TeX:UTF-8"

\chapter{}
\textbf{
Give an algorithm to detect whether a give undirected graph contains a cycle. If the graph contains a cycle, then your algorithm should output one. (It should not output all cycles in the graph, just one of them.) The running time of your algorithm should be $O(m+n)$ for a graph with $n$ nodes and $m$ edges.
}


\hspace*{\fill} \\

We design the algorithm based on depth-first traversal. As shown in Alg~\ref{find_cycle},we mark the difference between our cycle detected algorithm and DFS algorithm as blue. It is clearly that, when we find a repeatedly accessing vertex, there is a cycle in the graph.
\begin{algorithm}
\caption{Cycle Detected}
\label{find_cycle}
\begin{algorithmic}[1]
\REQUIRE $s$
\STATE Initialize $S$ to be a stack with one elements $s$
\WHILE{$S\neq \phi$}
\STATE{Take a node $u$ from $S$}
    \IF{Explored[$u$]=$false$}
        \STATE Set Explored[$u$]=$true$
        \FOR{each edge($u,v$) incident to $u$}
            \STATE{Add $v$ to the stack $S$}
        \ENDFOR
    \ELSE
        \STATE{\color{blue}Return TRUE}
    \ENDIF
\ENDWHILE
\STATE Return $FALSE$

\end{algorithmic}
\end{algorithm}

Because we only modify one step in the "if" judgement in the while cycle, it won't affect the time complexity of the original DFS time. Therefore, the time complexity is $O(m+n)$.