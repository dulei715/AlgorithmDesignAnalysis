% !Mode:: "TeX:UTF-8"

\chapter{}

2、假设你在2001年年中以35.55元的价格买入100股贵州茅台,买入后贵州茅台的股价每半年下跌10\%,期间你将收到分红全部再投入,并且一股不卖,那么傻持到2019年,你手中贵州茅台的市值会变成多少?计算年化收益率是多少?并根据结果谈谈你的理解! (年度分红均在次年年中到账;红利税一律为10\%,不考虑交易佣金;不考虑送转股对股价影响,计算结果四舍五入)
\hspace*{\fill} \\

{\kai
\begin{table}[H]
\caption{无脑复利投资贵州茅台}
\label{tb}
\centering
\scalebox{0.6}{

}
\end{table}

把数据输入到Excel表,插入``年数"数据方便统计。记2001H年为第0年,并用$T_0$表示,2001年为第0.5 年,并用$T_{0.5}$表示,以此类推。记股价为$P$,根据股价每半年跌10\%,可得 $P_{k+0.5}=0.9\times P_{k}$。设分红为$S$,则由表可知$S_0=0$,$S_{0.5}=6$以此类推。设送转增送股为$G$,则由表可知$G_0=0$,$G_{0.5}=1$,以此类推。设分红再投后持股量为$C$,则$C_0=C_{0.5}=100$,设收到的红利为$E$,则有$E_0=E_{0.5}=0$。设红利买入股为$N$,则有$N_0=N_{0.5}=0$,设转增股为$M$,则有$M_0=M_{0.5}=0$,设总市值为$Q$,则有$Q_0=3555$。设年化收益率为$R$,则有$R_0=0$。

红利收益$E_k=S_{k-0.5}\times 0.1+C_{k-1}\times(1-10\%)$; 红利买入股数 $N_k=\frac{P_k}{E_k}$;转增股$M_k=\frac{C_{k-1}}{10}\times G_{k-0.5}$;分红再投后持股量$C_{k+0.5}=C_{k}=C_{k-1}+N_{k-1}+M_{k-1}$($k$为整数);总市值$Q_k=P_k\times C_k$。由此我们能计算出每年的分红再投后持股量,收到的红利,红利买入股,转增股以及总市值。最后得到2019年贵州茅台的总市值接近2.25亿元。

根据复利公式
$$Q_0\times(1+R_T)^{T}=Q_T$$
可得
$$R_T=(\frac{Q_T}{Q_0})^{\frac{1}{T}}-1$$
由此可计算出到2019年,年化收益达到81.76\%。

贵州茅台属于好公司,好公司的股价下跌对投资者来说是机会。因为好公司会持续每年分红,将分红再以更低的价格购买股票,相当于以更低的价格获取了好公司更多的份额。从计算数据可知,将每年股息收入再投入购买贵州茅台股票,18.5 年平均年化收益率达到了81.76\%!即便市值以每半年10\%速度下降,总市值也在升高。一家公司的真实利润减少才是真正风险,如果投资一家好公司,它的股价跌了,我们完全不用担心,只要他的利润还在增长。因此我们要认真学好筛选好公司技能。


}
